%%
%% This is file `sample-manuscript.tex',
%% generated with the docstrip utility.
%%
%% The original source files were:
%%
%% samples.dtx  (with options: `all,proceedings,bibtex,manuscript')
%% 
%% IMPORTANT NOTICE:
%% 
%% For the copyright see the source file.
%% 
%% Any modified versions of this file must be renamed
%% with new filenames distinct from sample-manuscript.tex.
%% 
%% For distribution of the original source see the terms
%% for copying and modification in the file samples.dtx.
%% 
%% This generated file may be distributed as long as the
%% original source files, as listed above, are part of the
%% same distribution. (The sources need not necessarily be
%% in the same archive or directory.)
%%
%%
%% Commands for TeXCount
%TC:macro \cite [option:text,text]
%TC:macro \citep [option:text,text]
%TC:macro \citet [option:text,text]
%TC:envir table 0 1
%TC:envir table* 0 1
%TC:envir tabular [ignore] word
%TC:envir displaymath 0 word
%TC:envir math 0 word
%TC:envir comment 0 0
%%
%% The first command in your LaTeX source must be the \documentclass
%% command.
%%
%% For submission and review of your manuscript please change the
%% command to \documentclass[manuscript, screen, review]{acmart}.
%%
%% When submitting camera ready or to TAPS, please change the command
%% to \documentclass[sigconf]{acmart} or whichever template is required
%% for your publication.
%%
%%
\documentclass[acmsmall,nonacm]{acmart}
%%
%% \BibTeX command to typeset BibTeX logo in the docs
\AtBeginDocument{%
  \providecommand\BibTeX{{%
    Bib\TeX}}}

\settopmatter{printacmref=false} % removes ACM reference format block
\renewcommand\footnotetextcopyrightpermission[1]{} % removes copyright footnote
\pagestyle{plain} % optional: removes fancy headers

\onecolumn
    

%% Divines management information.  This information is sent to you
%% when you complete the rights form.  These commands have SAMPLE
%% values in them; it is your responsibility as an author to replace
%% the commands and values with those provided to you when you
%% complete the rights form.
\setcopyright{acmlicensed}
\copyrightyear{2025}
\acmYear{2025}
\acmDOI{XXXXXXX.XXXXXXX}
%% These commands are for a PROCEEDINGS abstract or paper.
\acmConference[Conference acronym 'XX]{N/A}{Jun 2025 - Dec 2025}{Woodstock, NY}
%%
%%  Uncomment \acmBooktitle if the title of the proceedings is different
%%  from ``Proceedings of ...''!
%%
%%\acmBooktitle{Woodstock '18: ACM Symposium on Neural Gaze Detection,
%%  June 03--05, 2018, Woodstock, NY}
\acmISBN{978-1-4503-XXXX-X/2018/06}


%%
%% Submission ID.
%% Use this when submitting an article to a sponsored event. You'll
%% receive a unique submission ID from the organizers
%% of the event, and this ID should be used as the parameter to this command.
%%\acmSubmissionID{123-A56-BU3}

%%
%% For managing citations, it is recommended to use bibliography
%% files in BibTeX format.
%%
%% You can then either use BibTeX with the ACM-Reference-Format style,
%% or BibLaTeX with the acmnumeric or acmauthoryear sytles, that include
%% support for advanced citation of software artefact from the
%% biblatex-software package, also separately available on CTAN.
%%
%% Look at the sample-*-biblatex.tex files for templates showcasing
%% the biblatex styles.
%%

%%
%% The majority of ACM publications use numbered citations and
%% references.  The command \citestyle{authoryear} switches to the
%% "author year" style.
%%
%% If you are preparing content for an event
%% sponsored by ACM SIGGRAPH, you must use the "author year" style of
%% citations and references.
%% Uncommenting
%% the next command will enable that style.
%%\citestyle{acmauthoryear}


%%
%% end of the preamble, start of the body of the document source.
\usepackage{amssymb}
\usepackage{amsmath}
\newcommand{\overbar}[1]{\mkern 1.5mu\overline{\mkern-1.5mu#1\mkern-1.5mu}\mkern 1.5mu}

% --- Safe cross-engine definition for \bo ---
\usepackage{iftex}

\ifPDFTeX
  \DeclareRobustCommand{\bo}{\mathfrak{B}}
\else
  \usepackage{fontspec}
  \newfontfamily{\vaifont}{Noto Sans Vai}

  % helper to typeset the glyph in text at the current math size
  \newcommand{\vaibo}[1]{{\vaifont #1}}

  \DeclareRobustCommand{\bo}{%
    \ifmmode
      \mathchoice
        {\text{\vaibo{ꕸ}}}            % displaystyle
        {\text{\vaibo{ꕸ}}}            % textstyle
        {\scriptstyle\text{\vaibo{ꕸ}}}% scriptstyle
        {\scriptscriptstyle\text{\vaibo{ꕸ}}}% scriptscriptstyle
    \else
      {\vaibo{ꕸ}}%
    \fi
  }
\fi

\newcommand{\bfa}{\bo}

\usepackage{subcaption}
\usepackage{tikz}

\usepackage[colorinlistoftodos]{todonotes}

% Define author-specific todos
\newcommand{\liam}[1]{\todo[inline, color=blue!40]{\begin{minipage}{\textwidth-4pt}\textbf{[Liam Monninger]:} #1\end{minipage}}}
\usetikzlibrary{cd}

\begin{document}

%%
%% The "title" command has an optional parameter,
%% allowing the author to define a "short title" to be used in page headers.
\title{Parabyzantine Protocols}

%%
%% The "author" command and its associated commands are used to define
%% the authors and their affiliations.
%% Of note is the shared affiliation of the first two authors, and the
%% "authornote" and "authornotemark" commands
%% used to denote shared contribution to the research.
\author{Liam Monninger}
\email{liam@ramate.io}
\affiliation{%
  \institution{Ramate LLC}
  \city{Durham}
  \state{California}
  \country{USA}
}

%%
%% By default, the full list of authors will be used in the page
%% headers. Often, this list is too long, and will overlap
%% other information printed in the page headers. This command allows
%% the author to define a more concise list
%% of authors' names for this purpose.
\renewcommand{\shortauthors}{Monninger}

%%
%% The abstract is a short summary of the work to be presented in the
%% article.
\begin{abstract}
   This memo is for friends and colleagues to explain the beginnings of my study of a category of distributed computing protocols which I call Parabyzantine. Below, I will begin with a motivation for the study. I'll then provide a trivial combinatorial example of a protocol assumed to be in the category. Finally, I'll suggest more abstract constructions with which the study is properly concerned.

    Most terms and notation follow [HKR]. The final section introduces provisional and more suggestive terms. 
\end{abstract}

%%
%% This command processes the author and affiliation and title
%% information and builds the first part of the formatted document.
\maketitle

\section{Motivation}
My motivation for studying Parabyzantine protocols, as I have dubbed them, is derived primarily from two phenomena in the world of distributed computing: the need for fault-tolerant order and compute inefficiencies of Byzantine fault-tolerant replica-based protocols.

\subsection{The General Need for Order}

In computing, we typically reason about partial functions. These are functions which deterministically map one value to another, but may not be defined for all inputs.

\begin{flalign*}
  &\hat{f}: S \to S \cup \{ \bot \} \text{ is a lifted partial function.} \\
  &f: S \rightharpoonup S \text{ is a partial function in its native category.} \\
\end{flalign*}

When composing these partial functions, we axiomatize that undefinedness propagates through the composition.

\begin{flalign*}
  &f, g: S \rightharpoonup S \\
  &f \circ g \text{ is defined if and only if} \\
  &\quad g(x) \text{ is defined and } \\
  &\quad f(g(x)) \text{ is defined} \\
\end{flalign*}

We find this to be a suitable model for programming modern machines, because we know how to design hardware which can logically represent functions with these properties and which is rarely corrupted by the surrounding environment.

However, for general computing tasks, we often have partial functions which are non-commutative. 

\begin{flalign*}
  &f \circ g \neq g \circ f \\
\end{flalign*}

A familiar example is subtraction on $\mathbb{N}$, which is naturally partial:

\begin{flalign*}
  &\mathrm{sub} : \mathbb{N} \times \mathbb{N} \rightharpoonup \mathbb{N} \\
  &\mathrm{sub}(a,b) =
  \begin{cases}
    a-b & \text{if } a \ge b,\\
    \text{undefined} & \text{otherwise.}
  \end{cases}
\end{flalign*}

And is also non-commutative:

\begin{flalign*}
  &\mathrm{sub}(5,3) = 2 \neq \mathrm{sub}(3,5) \text{ (undefined).}
\end{flalign*}

In general, for a family $f_1, \ldots, f_n$ of partial functions, the composite is determined by the ordered sequence $(f_1, \ldots, f_n)$ not the unordered set $\{ f_1, \ldots, f_n \}$. In local hardware cases---such as a single hart---we can often form reasonable assumptions or conventions, such that something like a state machine reading through an assembly is equivalent to a sequence of partial functions. In the context of general distributed computing tasks, however, we tend to make these assumptions more minimal. 

In distributed computing, we may instead refer to a set of processes $P$ which are able to share information with one another via messages $m \in M$ to varying degrees of success. Then, within this context, we hope to achieve results up to some notion of correctness and consistency, i.e., up to some notion of equivalence with the local hardware case.

Put more provocatively, a distributed computing protocol often has to decide from a chaotic world of messages what a reasonable order of partial functions should be. Signatures can be placed on messages to ensure they are authentic. Prover systems can be built to attest to the correctness of the value of a partial function. But, these may mean little without a reasonable order.

\subsection{2f + 1}

Often the success of a distributed computing protocol is framed in terms of its ability to induce agreement or consensus. And, in the abundant cases where order matters, this necessarily means consensus on said order. 

If we take set of processes $|P| = 3f + 1$ and ask for the unique messages $M_i \subseteq M$ which indicate computing the partial function at index $i$, we may wonder what different requirements on the agreement of these messages might mean. 

A simple majority $|M_i| > \frac{|P|}{2}$ would ensure by the Generalized Pigeonhole Principle that any two subsets of messages $M_{i}, M_{j}$ would intersect in at least one process. In a sense, at least one process can attest to each ordering step. By recurrence, this reads out a total order.

Let $Q: M \to 2^P$ be the function which maps a message to the set of processes which have sent it, representing a quorum. Then, we may define order $Order(M_i)$ as follows:

\begin{flalign*}
  &Q(M_i) = \bigsqcup_{m \in M_i} Q(m) \\
  &O(M_i, M_j) := (Q(M_i) \cap Q(M_j) \neq \emptyset) \\
  &Order(M_i) = Order(M_{i-1}) \land C(M_{i-1}, M_i) \\
  &\quad = Order(M_0) \land O(M_0, M_1) \land \cdots \land  O(M_{i-1}, M_i) \\
\end{flalign*}

Importantly, this simple majority model does not account for the possibility that a process may fault. As it turns out, the most aggressive assumption we can make and still retain the possibility of these intersecting sets amongst non-faulty processes is that at most $f$ processes may fault and quorums must be formed by at least $2f+1$ processes. This is the Byzantine fault model.

Let $H$ be the set of honest processes and $F$ be the set of faulty processes. Any two quora, under the Byzantine fault model, must then intersect in at least $f+1$ processes:

\begin{flalign*}
&|Q(M_1) \cap Q(M_2)|=\\
&\quad |Q(M_1)| + |Q(M_2)| - |Q(M_1) \cup Q(M_2)| \\
&\quad \geq (2f + 1) + (2f + 1) - (3f + 1)\\
&= f + 1 \\
\end{flalign*}

We can further deduce that any two quora must intersect in at least one honest process $h \in H$ in the presence of $f$ faulty processes:

\begin{flalign*}
  &|Q(M_1) \cap Q(M_2)|\\
  &\quad = |Q(M_1) \cap Q(M_2) \cap H| + |Q(M_1) \cap Q(M_2) \cap F|\\
  &|Q(M_1) \cap Q(M_2) \cap H|\\
  &\quad = |Q(M_1) \cap Q(M_2)| - |Q(M_1) \cap Q(M_2) \cap F|\\
  &f \geq |Q(M_1) \cap Q(M_2) \cap F|\\
  &|Q(M_1) \cap Q(M_2) \cap H| \geq f + 1 - f = 1\\
\end{flalign*}

A straightforward interpretation of this fact renders the ubiquitous family of Byzantine fault-tolerant state machine replication protocols (BFT SMR). Without a prover system---which is common in practice owing to the computational expense of systems like ZKPs---BFT SMRs require at least $2f+1$ of the honest processes to be engaged in the computing and broadcasting a partial function $f_i, f_{i+1}, \cdots, f_n$. With prover systems, this requirement holds up to the task of ordering. 

Roughly speaking, if a Byzantine fault-tolerant SMR protocol tolerates $f$ faulty processes, then it uses at best $\frac{1}{2f+1}$ of the available non-faulty compute.

\section{A Parabyzantine Protocol}

If we want to access more compute in a Byzantine fault-tolerant setting, an initial thought is to simply relax the requirement on order. Perhaps we can include fewer processes and allow our total order to fail with an acceptably low likelihood. Indeed, as I will show, even a fairly straightforward application of this concept yields a protocol which is computationally efficient and low loss. 

However, before we endeavor this protocol, let's first consider what computationally efficient and a low loss would mean.

As referenced above, SMR requires at least $2f+1$ processes to certify each partial function. Thus, each logical step costs $\Theta(2f + 1)$ replicated work. So, in abstract, if we use fewer than $2f+1$ processes, we will have improved compute efficiency.

What is meant by low loss may be less obvious. In the Byzantine fault-tolerant setting, we have described a discrete mathematical limit that tells us we cannot have a property we called $Order(M_i)$ without involving at least $2f+1$ processes. So, the notion that we can have protocol which is computationally more efficient seems to contradict this limit.

However, we may consider the following objective function which is minimized for $M_i$ when $Order(M_i) = 1$:

\begin{flalign*}
  &Obj(M_i) =  0 \iff Order(M_i) = 1\\ 
  &Obj(M_i) =  1 \iff Order(M_i) = 0\\ 
\end{flalign*}

In its discrete case, we cannot improve. But, if we constrain on the average case $E[Obj(M_i)] = Pr[Obj(M_i) = 1] \leq \epsilon$, we have much more flexibility.

In a sense, it is precisely this sort of assumption which gives rise to the name Parabyzantine. These are protocols which act like Byzantine fault-tolerant protocols up to some well-understood point. In the case of the \textsc{Resample} protocol, this point will be an average-case guarantee. While we are under the veil of average case, \textsc{Resample} will be Byzantine fault-tolerant. Once we remove the veil, \textsc{Resample} will not be. 

\subsection{The \textsc{Resample} Protocol}

We define the following Byzantine fault-allowing protocol $\text{\textsc{Resample}}$:

\begin{itemize}
  \item For a given index, $i$ on $M$, pick a random subcommittee $K_i \subseteq P$ s.t. $|K_i| = 3k + 1$.
  \item Accept $M_i$ if and only if $|Q_{K_i}(M_i)| \geq 2k + 1$. Otherwise, pick another random subcommittee and repeat.
\end{itemize}

Let...

\begin{flalign*}
  &f' = |K_i \cap F|, h' = |K_i \cap H|, f' + h' = |K_i| = 3k + 1
\end{flalign*}

Observe the following possible outcomes for selection of a random subcommittee:

\begin{itemize}
  \item $|K_i \cap H| \geq 2k + 1 \iff \text{ Divine } = D$, representing a subcommittee which has an honest supermajority which computes $M_i$ correctly.
  \item $|K_i \cap H| \leq k \iff \text{ Corrupt } = C$, representing a subcommittee which has a dishonest supermajority and may compute $M_i$ incorrectly.
  \item $k < |K_i \cap H| < 2k + 1 \iff \text{ Undecided } = U$, representing all other cases wherein neither an honest nor dishonest supermajority exists and may either compute $M_i$ correctly or disagree internally and not render a supermajority.
\end{itemize}

\subsection{The Loss of the \textsc{Resample} Protocol}

\begin{figure}[hbt!]
  \centering
  \includegraphics[width=\linewidth]{figures/pr_bad.png}
  \caption{$\Pr[B]$ representing the probability of selecting a corrupt subcommittee as a function of $f$ for different sampling ratios $\gamma = \frac{k}{f}$.}
  \label{fig:pr_bad}
\end{figure}

Let $\mathcal{S}(f, k)$ represent the total number of ways to select the subcommittee without replacement:

\begin{flalign*}
  &\mathcal{S}(f, k) = \binom{3f + 1}{3k + 1} \\
\end{flalign*}

The total number of ways to select a $\text{Corrupt}$ subcommittee is:

\begin{flalign*}
  &\mathcal{S}_{C}(f, k) = \sum_{f' = 2k + 1}^{\min(3k + 1, f)} \binom{f}{f'} \cdot \binom{2f + 1}{3k + 1 - f'} \\
\end{flalign*}

The total number of ways to select a $\text{Divine}$ subcommittee is:

\begin{flalign*}
  &\mathcal{S}_{D}(f, k) = \sum_{h' = 2k + 1}^{\min(3k + 1, 2f + 1)} \binom{2f + 1}{h'} \cdot \binom{f}{3k + 1 - h'} \\
\end{flalign*}

All other outcomes are by definition Undecided, so the total number of ways to select an Undecided subcommittee is:

\begin{flalign*}
  &\mathcal{S}_{U}(f, k) = \mathcal{S}(f, k) - \mathcal{S}_{C}(f, k) - \mathcal{S}_{D}(f, k) \\
\end{flalign*}

Let $S$ represent the transient state in which the \textsc{Resample} is sampling, i.e., has just started or produced a Undecided subcommittee. Let $G$ represent the absorbing state in which \textsc{Resample} makes a "good" decision to select a Divine subcommittee. Let $B$ represent the absorbing state in which \textsc{Resample} makes a "bad" decision to select a Corrupt subcommittee. The Markov Chain is then given by the state space $\Omega$ and transition matrix $T$.

\begin{flalign*}
&\Omega = \{S, G, B\} \\
&T = \begin{pmatrix}
  \Pr[U] & \Pr[D] & \Pr[C] \\
  0   & 1   & 0   \\
  0   & 0   & 1
\end{pmatrix} \\
\end{flalign*}

The probability of the algorithm making a "bad" decision to select a Corrupt subcommittee $\Pr[B]$ is then:

\begin{flalign*}
  &\text{Let } x := \Pr[\text{ eventually hitting B } | \text{ start in S }] \\
  &x = \Pr[C] \cdot 1 + \Pr[D] \cdot 0 + \Pr[U] \cdot x \\
  &x - \Pr[U] \cdot x = \Pr[C] \implies x \cdot (1 - \Pr[U]) = \Pr[C] \implies x = \frac{\Pr[C]}{1 - \Pr[U]} \\
  &\text{Observe that... } \\
  &1 - \Pr[U] = \Pr[D] + \Pr[C] \implies \Pr[B] = \frac{\Pr[C]}{\Pr[D] + \Pr[C]} \\
 \end{flalign*}

Conversely, the probability of the algorithm making a "good" decision to select a Divine subcommittee $\Pr[G]$ is then:

\begin{flalign*}
  &\Pr[G] = 1 - \Pr[B] = 1 - \frac{\Pr[C]}{\Pr[D] + \Pr[C]} = \frac{\Pr[D]}{\Pr[D] + \Pr[C]} \\
\end{flalign*}

As I will show, the probability $\Pr[C]$ drops off quickly for a fixed sampling ratio $\gamma = \frac{k}{f}$ (with, say, $k=\lfloor \gamma f \rfloor$). First observe that the number of faulty replicas in a sampled subcommittee follows a hypergeometric distribution:
\begin{flalign*}
  &X := |K_i \cap F| \sim \text{Hypergeometric}(N,K,n), \\
  &C \equiv \{X \ge 2k+1\}, \\
  &\Pr[C](\gamma,f) = \frac{\mathcal{S}_C\!\left(f,\,\lfloor\gamma f\rfloor\right)}{\mathcal{S}\!\left(f,\,\lfloor\gamma f\rfloor\right)}.
\end{flalign*}

Standard large-deviation bounds for sampling without replacement give a KL-based tail bound:
\begin{flalign*}
  &\Pr[X \ge qn] \le \exp\!\left(-n \cdot D(q \,\|\, p)\right), \\
  &D(q \,\|\, p) = q \log\!\left(\frac{q}{p}\right) + (1-q)\log\!\left(\frac{1-q}{1-p}\right).
\end{flalign*}

In our case $p=\frac{f}{3f+1}\approx \frac13$, $q=\frac{2k+1}{3k+1}\approx \frac23$, and $n=3k+1$. Hence
\begin{flalign*}
  &D\!\left(\frac23 \,\middle\|\, \frac13\right)
    = \frac23 \log 2 + \frac13 \log \frac12
    = \frac13 \log 2.
\end{flalign*}

Therefore,
\begin{flalign*}
  &\Pr[C] \lesssim \exp\!\left(-(3k+1)\cdot \frac13 \log 2\right)
   = 2^{-(3k+1)/3}, \\
  &\Pr[C](\gamma,f) \lesssim 2^{-(3\gamma f + 1)/3}
   = 2^{-\gamma f - 1/3}
   = 2^{-\Omega(f)}.
\end{flalign*}

In particular, for any fixed $\gamma\in(0,1)$,
\begin{flalign*}
  &\lim_{f \to \infty} \Pr[C](\gamma, f) = 0 \forall \gamma \in (0, 1) \\
\end{flalign*}

\subsection{The Cost of the \textsc{Resample} Protocol}

The expected cost of \textsc{Resample}, measured in the total number of processes engaged across all sampling attempts, remains on the order $\Theta(k)$, but admits a constant-factor improvement over the $2f+1$ quorum required by standard SMR.

Fix an index $i$ and write $\Pr[U] := \Pr[\mathcal{U}]$ for the probability that a uniformly sampled subcommittee of size $3k+1$ is \emph{Undecided}. Each sampling attempt terminates (i.e., yields a decision) precisely when the outcome is not $\mathcal{U}$. Thus, if $T$ denotes the number of sampling attempts until termination, then $T$ is geometric with success probability $1-\Pr[U]$:

\begin{flalign*}
  &\Pr[T = t] = \Pr[U]^{t-1}\cdot (1-\Pr[U]) \qquad \text{for } t \in \mathbb{N},\\
  &\mathbb{E}[T]
    = \sum_{t=1}^{\infty} t \cdot \Pr[U]^{t-1}(1-\Pr[U])
    = \frac{1}{1-\Pr[U]}.
\end{flalign*}

Each attempt engages at most $3k+1$ processes. Therefore the expected number of process-participations consumed per decision is

\begin{flalign*}
  &\mathbb{E}[\text{Processes Used}]
    = (3k+1)\cdot \mathbb{E}[T]
    = \frac{3k+1}{1-\Pr[U]}.
\end{flalign*}

In regimes of interest (e.g., for a fixed sampling ratio $\gamma=\tfrac{k}{f}$ with large $f$), $\Pr[U]$ is typically bounded away from $1$; in particular, if $\Pr[U]\approx \tfrac{1}{2}$, then

\begin{flalign*}
  &\mathbb{E}[T] \approx \frac{1}{1-\frac{1}{2}} = 2,\\
  &\mathbb{E}[\text{Processes Used}] \approx 2(3k+1) = \Theta(k).
\end{flalign*}

Thus, \textsc{Resample} is Parabyzantine in the sense described above: it replaces a worst-case quorum requirement with an average-case sampling cost, remaining computationally efficient while permitting a tunably low probability of failure.

\section{An Expanded View}


%%
%% The acknowledgments section is defined using the "acks" environment
%% (and NOT an unnumbered section). This ensures the proper
%% identification of the section in the article metadata, and the
%% consistent spelling of the heading.
\begin{acks}

\end{acks}

%%
%% The next two lines define the bibliography style to be used, and
%% the bibliography file.
\bibliographystyle{ACM-Reference-Format}
\bibliography{sample-base}


%%
%% If your work has an appendix, this is the place to put it.
\appendix

\end{document}
\endinput
%%
%% End of file `sample-manuscript.tex'.
